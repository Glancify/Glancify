\documentclass[12pt]{article}
\usepackage{hyperref}
\usepackage{graphicx}
\usepackage{array}
\usepackage{tabu}
\usepackage[table]{xcolor}
 

\setlength{\arrayrulewidth}{1mm}
\setlength{\tabcolsep}{18pt}
\renewcommand{\arraystretch}{1.6}
\renewcommand{\today}{October 7 , 2017}


\begin{document}

\begin{titlepage}
		\begin{center}
			
		%\vspace*{1cm}
			
			\LARGE{\textbf{System Test Plan}}
			
			\vspace{1.5cm}
			
			\textbf{Glancify}\\
			
			\small{Version 1.0}
			\vspace{3cm}
		 
            \large{Author: \hspace{0.5cm}Saurabh Srivastava \\\hspace{1cm}Sakshee Jain}	
    	
			\vspace{1.5cm}
			
			\today
			
			\vspace{4.5cm}
			\includegraphics[width=0.17\textwidth]{iiitv.png} \\
			\Large{Indian Institute of Information Technology Vadodara} \\
			
		\end{center}
	\end{titlepage}
	\newpage
	
	\begin{tabular}{ |m{4em}| m{10em}| m{10em}|}
	    \hline
	    Version & Author & Reviewer\\
	    \hline
	     1.0 &Saurabh, Sakshee& Himanshu, Deepak, Piyush, Neelansh \\
	     \hline
	\end{tabular}
	
	
	\newpage
\textbf{Team members :} \\
\begin{center}
	\begin{tabular}{ |m{10em} m{8em}m{9em}|}
		\hline
		TEAM MEMBER &   & ID        \\
		\hline
		Himanshu Singhal &   & 201551014 \\
		Piyush Sikarawal &   & 201551020 \\
		Saurabh Srivastava &   & 201551032 \\
	    Deepak Sandrana &   & 201551033 \\
		Sakshee Jain &   & 201551074 \\
		Neelansh Sahai &   & 201551086 \\ 
		\hline
	\end{tabular}
	
\end{center}

	\newpage
	\tableofcontents
	\newpage
	
	\section{Introduction}
    \subsection{Purpose}
	 Test plan is the project plan for the testing work to be done. It is a collection of test cases or a set of test procedures.This document describes the plan for testing the developed Glancify application system against the software requirements as defined in the Software Requirement Specification(SRS) document. The purpose of these system tests is to make sure that the software system developed during the project complies with the desired requirements, both functional and nonfunctional.
    
    \subsection{Overview of Product}
    We are designing Glancify application so that users are able to have a glance at their notifications for Social Networking Platforms they use. It will reflect the live user data for respective platform in form of tokens.
    
    \subsection{Target Audience}
    Target Audience for this document will be team members, and users/clients. We tested this in such a way that users/clients will not face any kind of security issues after release of the product to the customer.
    
    \section{Scope}
    The scope of this document is mainly to target the manual testing and validating data in report output as per Requirements Specifications provided.
    
    \subsection{Test items}
    We intend to test: client side application, we need to check that the application fulfils its objective and verify that all the features specified in the SRS are functional according to the expected result.
    
    \subsection{Features to be tested}
    \begin{itemize}
        \item Authentication to each SNP to be added
        \item Successful adding of each SNPs
        \item Creation of token for added SNP
        \item Deletion of token for removed SNP
        \item Direct to respective platform page
        \item Successful authentication response
        \item Un-Successful authentication error response
        \item Live user data fetch and display
        \item Empty/Error response 
    \end{itemize}
    
    \section{Approach}
    We intend to follow step by step procedure for the testing of Glancify application. First we will check the function of each module, and working of every component separately. After the successful testing of the working of each single module, we will integrate all the modules and test if all the modules and their interfaces are working in coordination. After that we will perform a system testing and see if the system as a whole is working correctly. Then we will check if the project is meeting the user acceptance requirements.
    
    \section{Test Strategy}
    It is an outline describing the testing approach of the software development cycle. This includes the testing objective, methods of testing new functions, total time and resources required for the project, and the testing environment.
    
    \subsection{Black box testing}
    Using Black Box testing, our focus will be on the functional requirements of the software.It will enable us to derive sets of input conditions that will fully exercise all functional requirements for a program without peering into its internal structures or workings.
    
    \subsection{Unit Testing}
    Unit Testing is a software testing method by which individual units of source code, sets of one or more computer program modules together with associated control data, usage procedures, and operating procedures, are tested to determine whether they are fit for use The purpose of unit testing is to confirm that each module works correctly as a standalone module.The units will be checked for both valid and invalid input whether they give an expected outcome or shows error messages correctly.
    
    \subsection{Integration Testing}
    This testing is basically used for testing if the Interfaces between different modules work correctly, when combined to form a group. Integration Testing is performed according to the Software Development Life Cycle (SDLC) after module and functional tests.
    
    \subsection{Regresssion Testing}
    Regression testing is a type of software testing that verifies that the software that was previously developed and tested still performs correctly after the units were interfaced with each other or it was changed and the bugs that were eliminated earlier do not reappear. The purpose of regression testing is to ensure that any kind of changes do not introduce new faults. One of the main reasons for regression testing is to determine whether a change in one part of the software affects other parts of the software.
    
    \subsection{System Testing}
    System testing is performed on the entire system in the context of a Functional Requirement Specification.The purpose of integration testing is to detect any inconsistencies between the software units that are integrated together.The aim of system testing is to ensure that the software product is made according to the requirements of the client and does indeed fulfil the intended purpose. This testing confirms that the entire software system as a whole works correctly and also all the desired functionality is present in the system that has been built.
    
    \subsection{User Acceptance testing}
    Acceptance testing will be done to confirm that the software system built satisfies the user defined acceptance criteria that would establish that the built software is acceptable by the client and is desirable to client. This may be done at the client site prior to actual handover of the software system.
    
    \section{Item- Pass/Fail criteria}
    This section specifies pass/fail criteria for the tests covered in this plan.An item will pass or fail based on the testing phase result.\\
    The pass criterion for our system are:
    \begin{itemize}
        \item The output of the feature is exactly the same as the expected output.
        
        \item For Authentication to each SNP feature, it should display readable error message for wrong user credentials and should login to that SNP for correct user credentials.
        
        \item Creation and Deletion of Token: It should create a space (Card) on web page to display data from the platform if client is adding a SNP. If client is removing the SNP, It should remove the card and re-adjust the available cards according to space on Web page.
        
        \item The link provided in each card must direct the user to their respective account of SNP.
        
        
        \item The content display as Notifications must be reliable, live and correct. The data can be successfully fetched from SNPs.
        
        \item For any kind of invalid inputs, it should warn the client and provide readable error messages.
    
    \end{itemize}
    If the above criteria are not met, the test team will assess the risk, identify actions with which it can be improved further and provide a recommendation.
    
    \section{Test Deliverables}
    Test Deliverables are the artifacts which are given to the stakeholders of software project during the software development lifecycle.\\The following are the test deliverables:
    \begin{itemize}
        \item Test Plan Document
        \item Test Cases
        \item Test Reports- covers system test results, problems, summary and analysis.
        \item Problem reports and corrective actions.
    \end{itemize}
    
    \section{Testing Tasks}
    The testing tasks mainly include writing test plan, creating test cases, conduct tests and evaluate results, and document test results.
    
    \section{Environmental Needs}
    The customer-side application, which is going to accessed over the Internet by the general public. The hardware required for testing the application will be any device that supports browsers.
    
    \section{Planning Risk}
    The plan for testing is subjected to change if any other testing methodology is found suitable for the project in addition to the methodologies currently selected for the project.Late delivery of the software, hardware or tools.Lack of availability of required resources like hardware, software, data or tools might lead to delay in the testing phase.
    
    
	
\end{document}