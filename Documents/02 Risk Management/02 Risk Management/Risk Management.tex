\documentclass[12pt]{article}
\usepackage{hyperref}
\usepackage{graphicx}
\usepackage{array}
\usepackage{tabu}
\usepackage[table]{xcolor}
 
\setlength{\arrayrulewidth}{1mm}
\setlength{\tabcolsep}{18pt}
\renewcommand{\arraystretch}{1.6}
\renewcommand{\today}{October 02, 2016}


\begin{document}

\begin{titlepage}
        \begin{center}
           
            %\vspace*{1cm}
           
            \LARGE{\textbf{Risk Management Report}}
           
            \vspace{1.5cm}
           
            \textbf{Glancify}\\
           
            \small{Version 1.0}
            \vspace{2cm}
         
           
           
            \vspace{1.5cm}
           
            \today
           
            \vspace{5cm}
            \includegraphics[width=0.17\textwidth]{iiitv.png} \\
            \Large{Indian Institute of Information Technology Vadodara} \\
           
        \end{center}
    \end{titlepage}
\newpage
\textbf{Team members :} \\
\begin{center}
	\begin{tabular}{ |m{10em} m{8em} m{9em}|}
		\hline
		TEAM MEMBER        &   & ID        \\
		\hline
		Himanshu Singhal             &   & 201551014 \\
		Piyush Sikarawal          &   & 201551020 \\
		Saurabh Srivastava              &   & 201551032 \\
	    Deeoak Sandrana     &   & 201551033 \\
		Sakshee Jain    &   & 201551074 \\
		Neelansh Sahai    &   & 201551086 \\ 
		\hline
	\end{tabular}
	
\end{center}
\newpage
\tableofcontents
	\newpage

\section{What is Risk?}
A risk is a potential problem.It might happen or it might not,this is uncertain.
We don’t know whether particular event will occur or not,but if it does,then definitely it has negative impact on project.So we need to maintain some strategies
to reduce these risks that we might encounter.

\section{Risk Management Paradigm}
\subsection{Identify}
Knowing the risk beforehand helps in avoiding any major wreck.
\subsection{Analyse}
Understand the nature of risk and gather information about the risk
\subsection{Plan}
Creating a risk-management plan
\subsection{Track}
Monitoring the status of risk
\subsection{Control}
Correct the deviations and make necessary amendments.
\subsection{Communicate}
Discussing about the emerging risks and current risks and steps to be taken
\section{ Risk Categorization}
\subsection{Negative Risks}
Any and every risk serves as a potential wreck for the solution or the team process in itself if not managed accordingly. Knowing about the potential risks beforehand serves a major deal to avoid any  issues since we can draft a mitigation plan for all such risks accordingly and avoid any mishap in future. 
\\
Thus any risk has the latent ability to beat the whole aim of software engineering i.e. a high quality cost effective solution.
\subsubsection{project risks}
\begin{enumerate}
    \item 
    \textbf{Not Meeting Deadline}
• Mitigation : Manage the work properly such that delay is minimum.
Constantly review the work done to check the lag time.

• Monitoring : Have clear idea of the dates and deadlines of each phase

and milestones of each phase. Check whether the assigned work is com-
pleted in time or not.

• Management : If work load is more on some sub group, members of
other sub group must coordinate with them in order to complete the
assigned work within given time.
    \item 
    \textbf{Staff turnover:} If any of my project team member is missing due to
some emergency situations(sick,drop-out from college.
\item 
\textbf{Too many decision makers:} 

\bigskip
\textbf{Description}
Among the 9 team members, each one
have different way of thinking and different way of implementation.

\bigskip
\textbf{Mitigation}
Each and every member of the team is responsible for
any thing in the project. Distribute the work uniformly, for small works

divide the team into sub groups. Maintain mutual understanding be-
tween the members.


\bigskip
\textbf{Monitoring}
Have clear understanding of goals to be achieved. re-
view the process with problems in mind, not the person. Work must

be distributed according to the capacity of each person.

\bigskip
\textbf{Management}
Discuss all the issues, however small with the whole
team. Don’t blame each other. If problem is not solved within the
team then consult project mentor or professor.

\item 
\textbf{Custom-coding to get features just right} Sometimes 90 percent of
a feature is good enough if delivering on that last 10 percent is going to
take 100s of hours of custom coding and testing and cause large time and
budget overruns.
\item 
\textbf{Custom-coding to get features just right} Sometimes 90 percent of
a feature is good enough if delivering on that last 10 percent is going to
take 100s of hours of custom coding and testing and cause large time and
budget overruns
\item 
\textbf{Changes in Requirements}
• Mitigation : Requirements must be defined only after conducting
proper survey. All the members of the team must go through the
requirements before preparing the SRS.
• Monitoring : The project plan and SRS are to be changed accordingly.
If we are in design phase we must chnge the design view etc.
• Management : The new requirements must be discussed with the
entire group. If it will take more time then the deadline and cost of the
project along with the project plan will change accordingly, ensuring
to complete the product as early as possible.
\end{enumerate}















\subsubsection{Technical risks}
\begin{enumerate}
\item
\textbf{SERVER CRASH}

\bigskip
\textbf{Description}
The extension collects and displays data using the API's(i.e data delivered to extension by corresponding platform API's). In this process whenever a particular networking platform faces a server crash (Webserver crash or Website crash like facebook or github or quora server crash) we will not be able to display that data for the corresponding platform.

\bigskip
\textbf{Management}
If the site server has crashed and the site is not opearational, the API's too would fail to retrieve data. We display a "HOST SERVER CRASH" message.

\item
\textbf{WEB CRASH}

\bigskip
\textbf{Description :}
The required methods and parameters may not be available in the particular platform API. The user may want a website or web platfrom to be included for which no API for data extraction is available.

\bigskip
\textbf{Mitigation :}
The extension very well lists beforehand which platforms it supports and lets user pick those platforms.

\bigskip
\textbf{Monitoring :}
We take user feedback and if in future the notifications for a platform is widely requested and the platform releases the corresponding API. We see the feasibility and develop it.   

\bigskip
\textbf{Management :}
If we receive a really good feedback for a particular  platform we can even mail and reach out to the technical team of the platform and see if getting API is possible. This may  be exclusive or free.

\item
\textbf{PRIVACY}

\bigskip
\textbf{Description}
One big obstacle may be the reluctance of users to share personal information with the platform. 

\bigskip
\textbf{Mitigation}
We do not collect any personal information of the users that is their login id and password and store it anywhere with our product.
We vouch for users privacy. Their data is as much secure with us as  with Google Chrome browser.

\bigskip
\textbf{Monitoring}
Through out coding phase we see specially for bugs that could inhibit user privacy. 

\bigskip
\textbf{Management}
We have injection testing in testing phase to destroy any source of leak in our system.
\item

\bigskip
\textbf{Delay or Error}

\bigskip
\textbf{Description}
In order to use the APIs we will first need to setup scripts to create queries on behalf of our extension with authorization based requests. Users will authenticate against our extension with their account credentials and we can then access their data with the resulting user access-token. There may come situations when there is delay or error in generation of these access-tokens due to invalid user-credentials or network error.

\bigskip
\textbf{Mitigation}
Internet connectivity has to be there for the extension to work properly.

\bigskip
\textbf{Monitoring}
If the user is logged in to a platform through the extension and he happens to change the login id and password through somewhere else, the extension loggs him out as and then.

\bigskip
\textbf{Management}
Login id and password and network errors are displayed on the extension as soon as any such issue arises.



\item
\textbf{LIMITED TIME DOMAIN}

\bigskip
\textbf{Description}
For some networking platforms the access-tokens are generated for a fixed amount of time means they have an expiring time like cookies. So if we are unable to perform data extraction within that period no data would be obtained of user.

\bigskip
\textbf{Mitigation :}
We use time efficient and tested API's. Most of these API's are tested by the source(i.e Facebook,Github,Twitter,Quora) at regular intervals and are completely reliable but we shall still choose the best ones around.

\bigskip
\textbf{Monitoring :}
Through out coding and unit testing we see for time stamps for data extraction process and  efficiency. 

\bigskip
\textbf{Management :}
Fast and light UI so that entire processing is used for getting API's to carry on with the extraction process.
\item
\bigskip

\textbf{Security}

\bigskip
\textbf{Description :}
Access-Tokens need to be stored somewhere (local/session storage or cookies) so any other malicious programmer with wrong intensions should not be able to access these access-tokens stored locally without the user's knowing.

\bigskip
\textbf{Mitigation :}
Access-Tokens storage specification stands laid at the beginning of the project depending upon the developers and their handling ease.

\bigskip
\textbf{Monitoring :}
The specification once laid confirms a stone's mark for others since security of the users will go for a toss if specification is not obeyed seriously. So we monitor the delivery of access tokens to the right hand.

\bigskip
\textbf{Management :}
In case of  any discrepancy the code must be checked and tokens directed to the right place since no software that hands users to the hands of hackers should be out in the market.
\subsubsection{Business risks}
We do not have a business client so business risk scope for the project stands really limited. We aim to develop a high quality solution for ourselves and our peers and if we succeed in doing so  we would consider our product doing good business. 
\end{enumerate}

\subsection{Positive Risks}
Positive risk is the chance that your objectives will produce too much of a good thing. Positive risks are uncertain, but favorable events if they have positively impact on the project objectives.
 Too positive risks are deemed as undesirable despite being positive at face value.These favorable opportunities tend to save cost and other resources of the project. Unlike Negative Risks, here your aim is to make this uncertain event happen.
 Positive risks may be referred to as opportunities.
 
 

We may find a senior or professional help working in same domain whose guidance can speed up our project completion an
 
\begin{itemize}
    \item Receiving so many signups for our new product that it crashes our websiteReceiving so many signups for our new product that it crashes our website
    \item Getting swamped by press requests because our project is so popular
\end{itemize}



Four Positive risk Response strategies to amplify the chances of the Positive risks:

\subsubsection{Exploit}
This strategy seeks to eliminate the uncertainty associated with a particular upside risk by ensuring the opportunity definitely happens.
\begin{enumerate}
    \item
    
    \textbf{New Technology }
    
    \textbf{Description}
    Experimenting of new technology may result in sparing 50percent of the development time
    
    \bigskip
    \textbf{Mitigation}
    
    
    \bigskip
    \textbf{Monitoring}
    
    
    \bigskip
    \textbf{Management}


\item
    \textbf{More Hits }
    \textbf{Description:}
    For example, experimenting of new technology may result in sparing 50percent of the development time
    
    \bigskip
    \textbf{Mitigation:}
    
    
    \bigskip
    \textbf{Monitoring:}
    
    
    \bigskip
    \textbf{Management:}
    
    \item if you want to get more hits on your new website, devote lots of time drumming up positive publicity by contacting journalists, writing press releases and getting your in-house communications team involved.
    
\end{enumerate}
\subsubsection{Share}
Sharing a positive risk involves allocating some or all of the ownership of the opportunity to a third party who is best able to capture the opportunity for the benefit of the project.
.
\begin{enumerate}

\item
\textbf{Overwhelming Response}

\bigskip
    \textbf{Description}
Lets say we get an overwhelming response and the current team falls short of either skill set or  working hands. 

    \bigskip
    \textbf{Mitigation}
We keep a record of skill set of people wishing to contribute as well as their skill set beforehand. We even designate multiple roles to group members if such a case arises    
    
    \bigskip
    \textbf{Monitoring}
If the risk arises beyond the level the team processes can handle we seek help from people in the records. We try to see this as an opportunity to ask for prospective investors. More money helps to bring in talent to share work with.    
    
    \bigskip
    \textbf{Management}
The project is uploaded on github and contributors would be welcome to play their part.    
    

\end{enumerate}
\subsubsection{Enhance}
The enhance strategy is used to increase the probability and/or the positive impacts of an opportunity.
\begin{enumerate}
\item 
\textbf{More Users}
We want to get more hits on our new website, and increase the number of users.

    \bigskip
    \textbf{Mitigation}
We associate a light and cool brand value with our app which takes care of users social interaction with ease userbase thus attracting more and more users.
   
    \bigskip
    \textbf{Monitoring}
   We devote lots of time drumming up positive publicity by contacting journalists, writing press releases and getting your in-house communications team involved.
  
    \bigskip
    \textbf{Management}
A regularly updated website with video tutorials of how the extension works and links to download and install the extension.    

\end{enumerate}
\subsubsection{Accept}
Accepting an opportunity is being willing to take advantage of the opportunity if it arises, but not actively pursuing it.
\begin{enumerate}
\item
\textbf{Press requests} 

\bigskip
\textbf{Description}
Getting attention of press towards the project idea.

\bigskip
\textbf{Management}
It can be a great opportunity for attracting investors and venture capitalists.
Press requests can be used for getting more customers for the 
product.
\item
\textbf{New Technology}

\bigskip
  \textbf{Description}
The adaptation of new technology within a module results in cost saving.

\bigskip
\textbf{Management}
Either the money can be used in the development phase or kept for later whenever cost issues arise in the project.
\end{enumerate}

\section{RMMM plan:}
\begin{itemize}
  \item  risk management strategy can be included in the software project plan or
the risk management steps can be organised into a separate risk
  \item mitigation,monitoring and management plan.
  \item RMMM plan documents all the work performed as part of risk analysis
and is used by the project leaded as part of the overall project plan
\end{itemize}
\section{Conclusion:}
\begin{itemize}
  \item  to manage the risks we need to establish a strong connection between the
user and client.Infact the developer must think as a user and identify the
risks he might face as a user when he was reviewing the technical risks
  \item software tools can he handful in identifying the risks and managing(Ex:RUP)
  \item Team need to have a strong base about risk management.
  \item Risk necessarily need not be in negative way and it can be viewed as an
opportunity to develop our projects in a better way
\end{itemize}

\end{document}
